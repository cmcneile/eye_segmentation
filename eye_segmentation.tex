%% 
%%  http://wso.williams.edu/wiki/index.php/LaTeX_Problem_Set_Template
%%

%%%%%%%%%%%%%%%%%%%%%%%%%%%%%%%%%%%%%%%%%%%%%%%
%%%This is a science homework template. Modify the preamble to suit
%%%your needs. 
%The junk text is   there for you to immediately see how the
%headers/footers look at first 
%typesetting.


\documentclass[12pt]{article}

%AMS-TeX packages
\usepackage{amssymb,amsmath,amsthm} 
%geometry (sets margin) and other useful packages
\usepackage[margin=1.25in]{geometry}
\usepackage{graphicx,ctable,booktabs}
\usepackage{url}
\usepackage{slashed}

%    
\begin{document}

\title{Segmentation of eye from MRI scans of heads}
%%\author{Craig McNeile}
\date{}

\maketitle


\section{Introduction}

%%  https://slicer.readthedocs.io/en/latest/user_guide/modules/segmentations.html
The eyes and lenses were segmentated from the MRI scans
using the 3D slicer software~\cite{kikinis20133d}.
The segmentation editor module~\cite{pinter2019polymorph}
in 3D slicer was used to manually
extract the eye and lenses.
The
files were then written to disk in STL format.

The stl files of the eyes are read into the python library,
and then converted into a point cloud.



\section{Fit models to the eye and lenses}

Navarro~\cite{navarro2009optical} reviews the different ways to
parameterise the retina and cornea.


\subsection{Ellipsoid fits}

A common function~\cite{van2021mri}
to parameter retina and lenses is the ellipsoid model.

The ellipsoid fit model is
\begin{equation}
\frac{(x-x_0)^2}{a^2} 
+
\frac{(y-y_0)^2}{b^2} 
+
\frac{(z-z_0)^2}{c^2}
= 1
\end{equation}
%%
where $x_0$ , $y_0$ , and $z_0$  are the center of the ellipsoid.
and  $a$, $b$ and $c$ are the length of the semi-axes.


Jie et al.~\cite{jia2004biconic} review the different ways to fit an
ellipsoid to set of points.

Use a non-linear optimizer in python to
minimze (Mean Square Error)
\begin{equation}
\mbox{MSE} = \frac{1}{n} \sum_{i=1}^n (f(x_i,y_i,z_i) - 1)^2
\end{equation}
%%
where $n$ are the number points and
$f$ is the left hand side of the ellipsoid equation.



  %%  https://en.wikipedia.org/wiki/Broyden%E2%80%93Fletcher%E2%80%93Goldfarb%E2%80%93Shanno_algorithm
  The minimization algorithm is  quasi-Newton method of Broyden,
  Fletcher, Goldfarb, and Shanno (BFGS). This uses derivative
  information of the loss function. The derivative is computed using
  a library called jax~\cite{bradbury2021jax},
  which computes the derivative from the python
  function.

  The errors on the parameters are estimated using the bootstrap
method~\cite{efron1992bootstrap}.

 \subsection{Biconic surfaces}


 The following model is suggested~\cite{burek1993mathematical}
  is a more general fit model than the ellipsoid.
 
\begin{equation}
  z^{model}  = \frac{c_x x^2 + c_y y^2}
  {1 +  \sqrt{ 1 - (1 + Q_x) c_x^2 x^2  - (1 + Q_y) c_y^2 y^2   }   }
 \end{equation}
%%
where $c_x$, $c_y$, $Q_x$ and $Q_y$  are parameters.
The $c_x$ and  $c_y$ parameters are the curvatures.
The parameters $Q_x$ and $Q_y$ are the conic constants.

The parameters of the Ellipsoid can be extracted from the 
\begin{eqnarray}
  R_x & = & \frac{a^2}{c}   \\
  R_y & = & \frac{b^2}{c}   \\
  Q_x & = & \frac{a^2}{c^2} - 1 \\
  Q_y & = & \frac{b^2}{c^2} - 1 
\end{eqnarray}
  
The parameters are determined by minimizing the mean square error.
\begin{equation}
\mbox{MSE} = \frac{1}{n} \sum_{i=1}^n (z^{model}_i - z_i )^2
\end{equation}


A more general model can be introduced
\begin{equation}
z^{model} = S(x,y) = b(x,y) + \sum_{k} c_k P_k(x,y) 
\label{eq:gen}
\end{equation}
%%
where $b(x,y)$ is the starting model and
$P_k(x,y)$ are polynomials labelled by $k$.
%%
In Optics one common sequence of polynomials used are the
Zernike polynomials~\cite{iskander2001optimal}.
Zernike polynomials are a sequence of polynomials
that are orthogonal on the unit disk (see the review~\cite{niu2022zernike}). 

The general model in equation~\ref{eq:gen} was
applied to fitting the Corneas of 123 subjects.

\section{Conclusion}

\bibliographystyle{h-physrev5}
\bibliography{eye}


\end{document}
