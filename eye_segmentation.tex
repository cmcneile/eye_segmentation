%% 
%%  http://wso.williams.edu/wiki/index.php/LaTeX_Problem_Set_Template
%%

%%%%%%%%%%%%%%%%%%%%%%%%%%%%%%%%%%%%%%%%%%%%%%%
%%%This is a science homework template. Modify the preamble to suit
%%%your needs. 
%The junk text is   there for you to immediately see how the
%headers/footers look at first 
%typesetting.


\documentclass[12pt]{article}

%AMS-TeX packages
\usepackage{amssymb,amsmath,amsthm} 
%geometry (sets margin) and other useful packages
\usepackage[margin=1.25in]{geometry}
\usepackage{graphicx,ctable,booktabs}
\usepackage{url}
\usepackage{slashed}

%    
\begin{document}

\title{Segmentation of eye from MRI scans of heads}
%%\author{Craig McNeile}
\date{}

\maketitle


\section{Introduction}

%%  https://slicer.readthedocs.io/en/latest/user_guide/modules/segmentations.html
The eyes and lenses were segmentated from the MRI scans
using the 3D slicer software~\cite{kikinis20133d}.
The segmentation editor module~\cite{pinter2019polymorph}
in 3D slicer was used to manually
extract the eye and lenses.
The
files were then written to disk in STL format.
We want to automate this processs, so that the eyes are extracted from
the MRI scans using machine learning.


\section{Automating the segmentation}

We followed the example scripts written by
Mokhtari Mohammed El Amine that used segmentation to extract
livers from MRI scans. The python code was available
on github \url{https://github.com/amine0110/Liver-Segmentation-Using-Monai-and-PyTorch}.

The example python script was built on the following libraries:

\begin{description}

  \item[monai] is a set of open-source, freely available collaborative frameworks built for accelerating research and clinical collaboration in Medical Imaging.  \url{https://monai.io/}

\item[pytorch] Pytorch~\cite{paszke2019pytorch}
  is an open source machine learning library  (\url{https://pytorch.org/}.)
    
\end{description}

\section{Conclusion}

\bibliographystyle{h-physrev5}
\bibliography{eye}


\end{document}
